%启动latex源程序
\documentclass[UTF-8]{ctexart}
%调用宏包
%geometry:利用 geometry 可以很方便的设置页面的大小。由于可以自动居中排放页面,自动计算并平衡页面各部分如页眉、页脚、左右边空等的大小,所以只需给出很少的信息就能得到满意的页面。
%AMSLaTeX:作为 AMSTeX 在 LaTeX 中地实现,AMSLaTeX 包括两部分,一是amsmath宏包,主要的目的是用来排版数学符号和公式,其中专门有 amsthm 宏包,提供对定理的排版。另一部分是amscls,提供了美国数学会要求的论文和书籍的格式。
%listings:排版 C, C++, Pascal 等源代码,提供语法加亮显示的功能。
%curves:不需要太多的TeXmemory,就能得到各种具有连续角度的曲线,包括 bezrer 曲线,虚线等。
%fancyhdr:用 fancyhdr 来设置页眉和页脚十分方便,而且可以在配合 CCT、CJK来设置中文的页眉等。
%graphics:插入图片所需宏包
\usepackage{amsmath} \usepackage{graphicx} \usepackage{amsmath} \usepackage{type1cm} \usepackage{lipsum} \usepackage{lastpage}
\usepackage{geometry} \usepackage{fancyhdr}




%预处理
\lhead{GYC\#20001019}
\rhead{page \thepage{} of 4 Pages}
\pagestyle{fancy}
\geometry{papersize={21cm,29.7cm}}
\geometry{left=2.5cm,right=2.5cm,top=2.5cm,bottom=2.5cm}
\cfoot{\thepage}



%开始论文撰写
\begin{document}
\title{15A论文分析}
\author{葛雨辰}
\date{\today}
\maketitle



\begin{abstract}         %摘要部分
\small{这是一篇关于葛雨辰成才的小论文。}
\end{abstract}




%newcommand处理操作
\newcommand{\song}{\CJKfamily{song}}       % 宋体
\newcommand{\fs}{\CJKfamily{fs}}           % 仿宋体
\newcommand{\kai}{\CJKfamily{kai}}         % 楷体
\newcommand{\hei}{\CJKfamily{hei}}         % 黑体
\newcommand{\li}{\CJKfamily{li}}           % 隶书
\newcommand{\yihao}{\fontsize{26pt}{36pt}\selectfont}          
% 一号, 1.4 倍行距
\newcommand{\erhao}{\fontsize{22pt}{28pt}\selectfont}          
% 二号, 1.25倍行距
\newcommand{\xiaoer}{\fontsize{18pt}{18pt}\selectfont}        
% 小二, 单倍行距
\newcommand{\sanhao}{\fontsize{16pt}{24pt}\selectfont}        
% 三号, 1.5倍行距
\newcommand{\xiaosan}{\fontsize{15pt}{22pt}\selectfont}        
% 小三, 1.5倍行距
\newcommand{\sihao}{\fontsize{14pt}{21pt}\selectfont}          
% 四号, 1.5 倍行距
\newcommand{\banxiaosi}{\fontsize{13pt}{19.5pt}\selectfont}    
% 半小四, 1.5倍行距
\newcommand{\xiaosi}{\fontsize{12pt}{18pt}\selectfont}        
% 小四, 1.5倍行距
\newcommand{\dawuhao}{\fontsize{11pt}{11pt}\selectfont}        
% 大五号, 单倍行距
\newcommand{\wuhao}{\fontsize{10.5pt}{15.75pt}\selectfont}     
% 五号, 单倍行距 




\tableofcontents
\section{2020新年快乐}
	\heiti\sanhao{2019寒假初作:15A论文分析}
	\songti\xiaosi{葛雨辰}
	\songti\xiaosi{\today}
	\subsection{未来科学家}
		\subsubsection{数学家葛雨辰}
			\paragraph{1}
				\lipsum[1]
					\subparagraph{2}
						\lipsum[1]
		\subsubsection{物理学家葛雨辰}
			\paragraph{3}
				\lipsum[2] 
		\subsubsection{化学家葛雨辰}
			\paragraph{4}
				\lipsum[3]
		\subsubsection{材料学家葛雨辰}
			\paragraph{5}
				\lipsum[4]




\begin{center}
	\huge{day 2020.1.26}
\end{center}

\begin{equation}
	bab^{-1}\in{H} \quad \frac{1}{2} \quad \sqrt{x}
\end{equation}

\begin{align}
&a = b+c+d \\
&x = y+z
\end{align}



\begin{center}
	{Table: Notation}
	\end{center}
	\begin{center}
		\begin{tabular}{*3{c}}
		\hline
		Abbreviation & Description\\
		\hline
		$E_{avg}$ & 选定规模处理厂单位处理能耗 \\
		$E_{sub}$ & 活性炭吸附工艺单位处理能耗容许值 \\
		$E_{tar}$ & 活性炭吸附工艺单位处理能耗目标值 \\
		\hline
		\end{tabular}

		$$E_{save}=E_{avg}-\frac{E_{tar}+E_{sub}}{2}}$$
	\end{center}




















\begin{thebibliography}{99}   
	\bibitem
		{shiye}失野健太郎.几何的有名定理.上海科学技术出版社,1986.                  
	\bibitem{quanjing}
		曲安金.商高、赵爽与刘辉关于勾股定理的证明.数学传播, 20(3), 1998. 

	\bibitem{Kline}克莱因.古今数学思想.上海科学技术出版社,2002.
\end{thebibliography}




\begin{appendix}            
	\section{附录}
	\songti\xiaosi 数学建模
\end{appendix}




\end{document}